% Options for packages loaded elsewhere
\PassOptionsToPackage{unicode}{hyperref}
\PassOptionsToPackage{hyphens}{url}
%
\documentclass[
]{article}
\usepackage{lmodern}
\usepackage{amssymb,amsmath}
\usepackage{ifxetex,ifluatex}
\ifnum 0\ifxetex 1\fi\ifluatex 1\fi=0 % if pdftex
  \usepackage[T1]{fontenc}
  \usepackage[utf8]{inputenc}
  \usepackage{textcomp} % provide euro and other symbols
\else % if luatex or xetex
  \usepackage{unicode-math}
  \defaultfontfeatures{Scale=MatchLowercase}
  \defaultfontfeatures[\rmfamily]{Ligatures=TeX,Scale=1}
\fi
% Use upquote if available, for straight quotes in verbatim environments
\IfFileExists{upquote.sty}{\usepackage{upquote}}{}
\IfFileExists{microtype.sty}{% use microtype if available
  \usepackage[]{microtype}
  \UseMicrotypeSet[protrusion]{basicmath} % disable protrusion for tt fonts
}{}
\makeatletter
\@ifundefined{KOMAClassName}{% if non-KOMA class
  \IfFileExists{parskip.sty}{%
    \usepackage{parskip}
  }{% else
    \setlength{\parindent}{0pt}
    \setlength{\parskip}{6pt plus 2pt minus 1pt}}
}{% if KOMA class
  \KOMAoptions{parskip=half}}
\makeatother
\usepackage{xcolor}
\IfFileExists{xurl.sty}{\usepackage{xurl}}{} % add URL line breaks if available
\IfFileExists{bookmark.sty}{\usepackage{bookmark}}{\usepackage{hyperref}}
\hypersetup{
  pdftitle={500\_83\_get\_mixed\_effect\_model\_results.R},
  pdfauthor={paul},
  hidelinks,
  pdfcreator={LaTeX via pandoc}}
\urlstyle{same} % disable monospaced font for URLs
\usepackage[margin=1in]{geometry}
\usepackage{color}
\usepackage{fancyvrb}
\newcommand{\VerbBar}{|}
\newcommand{\VERB}{\Verb[commandchars=\\\{\}]}
\DefineVerbatimEnvironment{Highlighting}{Verbatim}{commandchars=\\\{\}}
% Add ',fontsize=\small' for more characters per line
\usepackage{framed}
\definecolor{shadecolor}{RGB}{248,248,248}
\newenvironment{Shaded}{\begin{snugshade}}{\end{snugshade}}
\newcommand{\AlertTok}[1]{\textcolor[rgb]{0.94,0.16,0.16}{#1}}
\newcommand{\AnnotationTok}[1]{\textcolor[rgb]{0.56,0.35,0.01}{\textbf{\textit{#1}}}}
\newcommand{\AttributeTok}[1]{\textcolor[rgb]{0.77,0.63,0.00}{#1}}
\newcommand{\BaseNTok}[1]{\textcolor[rgb]{0.00,0.00,0.81}{#1}}
\newcommand{\BuiltInTok}[1]{#1}
\newcommand{\CharTok}[1]{\textcolor[rgb]{0.31,0.60,0.02}{#1}}
\newcommand{\CommentTok}[1]{\textcolor[rgb]{0.56,0.35,0.01}{\textit{#1}}}
\newcommand{\CommentVarTok}[1]{\textcolor[rgb]{0.56,0.35,0.01}{\textbf{\textit{#1}}}}
\newcommand{\ConstantTok}[1]{\textcolor[rgb]{0.00,0.00,0.00}{#1}}
\newcommand{\ControlFlowTok}[1]{\textcolor[rgb]{0.13,0.29,0.53}{\textbf{#1}}}
\newcommand{\DataTypeTok}[1]{\textcolor[rgb]{0.13,0.29,0.53}{#1}}
\newcommand{\DecValTok}[1]{\textcolor[rgb]{0.00,0.00,0.81}{#1}}
\newcommand{\DocumentationTok}[1]{\textcolor[rgb]{0.56,0.35,0.01}{\textbf{\textit{#1}}}}
\newcommand{\ErrorTok}[1]{\textcolor[rgb]{0.64,0.00,0.00}{\textbf{#1}}}
\newcommand{\ExtensionTok}[1]{#1}
\newcommand{\FloatTok}[1]{\textcolor[rgb]{0.00,0.00,0.81}{#1}}
\newcommand{\FunctionTok}[1]{\textcolor[rgb]{0.00,0.00,0.00}{#1}}
\newcommand{\ImportTok}[1]{#1}
\newcommand{\InformationTok}[1]{\textcolor[rgb]{0.56,0.35,0.01}{\textbf{\textit{#1}}}}
\newcommand{\KeywordTok}[1]{\textcolor[rgb]{0.13,0.29,0.53}{\textbf{#1}}}
\newcommand{\NormalTok}[1]{#1}
\newcommand{\OperatorTok}[1]{\textcolor[rgb]{0.81,0.36,0.00}{\textbf{#1}}}
\newcommand{\OtherTok}[1]{\textcolor[rgb]{0.56,0.35,0.01}{#1}}
\newcommand{\PreprocessorTok}[1]{\textcolor[rgb]{0.56,0.35,0.01}{\textit{#1}}}
\newcommand{\RegionMarkerTok}[1]{#1}
\newcommand{\SpecialCharTok}[1]{\textcolor[rgb]{0.00,0.00,0.00}{#1}}
\newcommand{\SpecialStringTok}[1]{\textcolor[rgb]{0.31,0.60,0.02}{#1}}
\newcommand{\StringTok}[1]{\textcolor[rgb]{0.31,0.60,0.02}{#1}}
\newcommand{\VariableTok}[1]{\textcolor[rgb]{0.00,0.00,0.00}{#1}}
\newcommand{\VerbatimStringTok}[1]{\textcolor[rgb]{0.31,0.60,0.02}{#1}}
\newcommand{\WarningTok}[1]{\textcolor[rgb]{0.56,0.35,0.01}{\textbf{\textit{#1}}}}
\usepackage{graphicx}
\makeatletter
\def\maxwidth{\ifdim\Gin@nat@width>\linewidth\linewidth\else\Gin@nat@width\fi}
\def\maxheight{\ifdim\Gin@nat@height>\textheight\textheight\else\Gin@nat@height\fi}
\makeatother
% Scale images if necessary, so that they will not overflow the page
% margins by default, and it is still possible to overwrite the defaults
% using explicit options in \includegraphics[width, height, ...]{}
\setkeys{Gin}{width=\maxwidth,height=\maxheight,keepaspectratio}
% Set default figure placement to htbp
\makeatletter
\def\fps@figure{htbp}
\makeatother
\setlength{\emergencystretch}{3em} % prevent overfull lines
\providecommand{\tightlist}{%
  \setlength{\itemsep}{0pt}\setlength{\parskip}{0pt}}
\setcounter{secnumdepth}{-\maxdimen} % remove section numbering

\title{500\_83\_get\_mixed\_effect\_model\_results.R}
\author{paul}
\date{2020-01-31}

\begin{document}
\maketitle

{
\setcounter{tocdepth}{3}
\tableofcontents
}
\begin{Shaded}
\begin{Highlighting}[]
\CommentTok{\# }
\end{Highlighting}
\end{Shaded}

\hypertarget{preamble}{%
\section{Preamble}\label{preamble}}

This script version tests the influence of environmental data and the
influence of voyages against biological responses. This script only
considers port in between which routes are present, such ports are fewer
the ports which have environmental distances available. (All ports with
measurements have environmental distances available.)

This script needs all R scripts named \texttt{500\_*.R} to have run
successfully, apart from
\texttt{/Users/paul/Documents/CU\_combined/Github/500\_05\_UNIFRAC\_behaviour.R}
It should then be called using a shell script. It will only accept
certain files currently, and otherwise abort. For further information
understand section Environment preparation. Also check
\texttt{/Users/paul/Documents/CU\_combined/Github/210\_get\_mixed\_effect\_model\_results.sh}

This code commentary is included in the R code itself and can be
rendered at any stage using
\texttt{rmarkdown::render\ ("/Users/paul/Documents/CU\_combined/Github/500\_83\_get\_mixed\_effect\_model\_results.R",\ clean\ =\ TRUE)}.
Please check the session info at the end of the document for further
notes on the coding environment.

\hypertarget{environment-preparation}{%
\section{Environment preparation}\label{environment-preparation}}

Empty buffer.

\begin{Shaded}
\begin{Highlighting}[]
\KeywordTok{rm}\NormalTok{(}\DataTypeTok{list=}\KeywordTok{ls}\NormalTok{())}
\end{Highlighting}
\end{Shaded}

Load Packages

\begin{Shaded}
\begin{Highlighting}[]
\KeywordTok{library}\NormalTok{ (}\StringTok{"tidyverse"}\NormalTok{) }\CommentTok{\# dplyr and friends}
\KeywordTok{library}\NormalTok{ (}\StringTok{"ggplot2"}\NormalTok{)   }\CommentTok{\# for ggCaterpillar}
\CommentTok{\# library ("ggbiplot")  \# better PCoA plotting, get via \textasciigrave{}library(devtools); install\_github("vqv/ggbiplot")\textasciigrave{}}
                      \CommentTok{\# uses \textasciigrave{}plyr\textasciigrave{} and needs to be loaded before \textasciigrave{}dplyr\textasciigrave{} in \textasciigrave{}tidyverse\textasciigrave{} }
\KeywordTok{library}\NormalTok{ (}\StringTok{"gdata"}\NormalTok{)     }\CommentTok{\# matrix functions}
\KeywordTok{library}\NormalTok{ (}\StringTok{"reshape2"}\NormalTok{)  }\CommentTok{\# melting}
\KeywordTok{library}\NormalTok{ (}\StringTok{"lme4"}\NormalTok{)      }\CommentTok{\# mixed effect model}
\KeywordTok{library}\NormalTok{ (}\StringTok{"sjPlot"}\NormalTok{)    }\CommentTok{\# mixed effect model {-} with plotting}

\KeywordTok{library}\NormalTok{ (}\StringTok{"vegan"}\NormalTok{)     }\CommentTok{\# metaMDS}
\KeywordTok{library}\NormalTok{ (}\StringTok{"cowplot"}\NormalTok{)   }\CommentTok{\# exporting ggplots}
\end{Highlighting}
\end{Shaded}

Functions

\begin{Shaded}
\begin{Highlighting}[]
\CommentTok{\# Loaded from helper script:}
\KeywordTok{source}\NormalTok{(}\StringTok{"/Users/paul/Documents/CU\_combined/Github/500\_00\_functions.R"}\NormalTok{)}
\end{Highlighting}
\end{Shaded}

\hypertarget{model-definitions}{%
\section{Model definitions}\label{model-definitions}}

\hypertarget{define-full-models}{%
\subsection{Define full models}\label{define-full-models}}

following
\texttt{https://stackoverflow.com/questions/25312818/using-lapply-to-fit-multiple-model-how-to-keep-the-model-formula-self-contain}

\begin{Shaded}
\begin{Highlighting}[]
\NormalTok{full\_formulae <{-}}\StringTok{ }\KeywordTok{list}\NormalTok{(}
  \CommentTok{\# Original formula used by Paul:}
  \CommentTok{\# Unifrac \textasciitilde{} Voyage counts  + env similarity + ecoregion + random port effects}
  \KeywordTok{as.formula}\NormalTok{(RESP\_UNIFRAC }\OperatorTok{\textasciitilde{}}\StringTok{ }\NormalTok{PRED\_TRIPS }\OperatorTok{+}\StringTok{ }\NormalTok{PRED\_ENV }\OperatorTok{+}\StringTok{ }\NormalTok{ECO\_DIFF }\OperatorTok{+}\StringTok{ }\NormalTok{(}\DecValTok{1} \OperatorTok{|}\StringTok{ }\NormalTok{PORT) }\OperatorTok{+}\StringTok{ }\NormalTok{(}\DecValTok{1} \OperatorTok{|}\StringTok{ }\NormalTok{DEST)),}

  \CommentTok{\# Original formula adjusted with Mandana\textquotesingle{}s frequencies:}
  \CommentTok{\# Unifrac \textasciitilde{} Voyage frequencies + env similarity + ecoregion + random port effects}
  \KeywordTok{as.formula}\NormalTok{(RESP\_UNIFRAC }\OperatorTok{\textasciitilde{}}\StringTok{ }\NormalTok{VOY\_FREQ }\OperatorTok{+}\StringTok{ }\NormalTok{PRED\_ENV }\OperatorTok{+}\StringTok{ }\NormalTok{ECO\_DIFF }\OperatorTok{+}\StringTok{ }\NormalTok{(}\DecValTok{1} \OperatorTok{|}\StringTok{ }\NormalTok{PORT) }\OperatorTok{+}\StringTok{ }\NormalTok{(}\DecValTok{1} \OperatorTok{|}\StringTok{ }\NormalTok{DEST)),}

  \CommentTok{\# Formulas from Word document with Mandana\textquotesingle{}s ballast risk estimates:}
  \CommentTok{\# Unifrac \textasciitilde{} Ballast FON shipping + env similarity + ecoregion + random port effects}
  \KeywordTok{as.formula}\NormalTok{(RESP\_UNIFRAC }\OperatorTok{\textasciitilde{}}\StringTok{ }\NormalTok{B\_FON\_NOECO }\OperatorTok{+}\StringTok{ }\NormalTok{PRED\_ENV }\OperatorTok{+}\StringTok{ }\NormalTok{ECO\_DIFF }\OperatorTok{+}\StringTok{ }\NormalTok{(}\DecValTok{1} \OperatorTok{|}\StringTok{ }\NormalTok{PORT) }\OperatorTok{+}\StringTok{ }\NormalTok{(}\DecValTok{1} \OperatorTok{|}\StringTok{ }\NormalTok{DEST)),}

  \CommentTok{\# Unifrac \textasciitilde{} Ballast HON shipping + env similarity + ecoregion + random port effects}
  \KeywordTok{as.formula}\NormalTok{(RESP\_UNIFRAC }\OperatorTok{\textasciitilde{}}\StringTok{ }\NormalTok{B\_HON\_NOECO }\OperatorTok{+}\StringTok{ }\NormalTok{PRED\_ENV }\OperatorTok{+}\StringTok{ }\NormalTok{ECO\_DIFF }\OperatorTok{+}\StringTok{ }\NormalTok{(}\DecValTok{1} \OperatorTok{|}\StringTok{ }\NormalTok{PORT) }\OperatorTok{+}\StringTok{ }\NormalTok{(}\DecValTok{1} \OperatorTok{|}\StringTok{ }\NormalTok{DEST)),}

  \CommentTok{\# Unifrac \textasciitilde{} Ballast FON risk* + \textasciitilde{}\textasciitilde{}ecoregion\textasciitilde{}\textasciitilde{} / env similarity + random port effects}
  \KeywordTok{as.formula}\NormalTok{(RESP\_UNIFRAC }\OperatorTok{\textasciitilde{}}\StringTok{ }\NormalTok{B\_FON\_SMECO }\OperatorTok{+}\StringTok{ }\NormalTok{PRED\_ENV }\OperatorTok{+}\StringTok{ }\NormalTok{(}\DecValTok{1} \OperatorTok{|}\StringTok{ }\NormalTok{PORT) }\OperatorTok{+}\StringTok{ }\NormalTok{(}\DecValTok{1} \OperatorTok{|}\StringTok{ }\NormalTok{DEST)),}

  \CommentTok{\# Unifrac \textasciitilde{} Ballast HON risk* + \textasciitilde{}\textasciitilde{}ecoregion\textasciitilde{}\textasciitilde{} / env similarity + random port effects}
  \KeywordTok{as.formula}\NormalTok{(RESP\_UNIFRAC }\OperatorTok{\textasciitilde{}}\StringTok{ }\NormalTok{B\_HON\_SMECO }\OperatorTok{+}\StringTok{ }\NormalTok{PRED\_ENV }\OperatorTok{+}\StringTok{ }\NormalTok{(}\DecValTok{1} \OperatorTok{|}\StringTok{ }\NormalTok{PORT) }\OperatorTok{+}\StringTok{ }\NormalTok{(}\DecValTok{1} \OperatorTok{|}\StringTok{ }\NormalTok{DEST))}
\NormalTok{)}
\end{Highlighting}
\end{Shaded}

\hypertarget{define-null-models}{%
\subsection{Define null models}\label{define-null-models}}

For Anova comparison. Order \emph{must} be the same as in list
\texttt{full\_models}.

\begin{Shaded}
\begin{Highlighting}[]
\NormalTok{null\_formulae <{-}}\StringTok{ }\KeywordTok{list}\NormalTok{(}
  \CommentTok{\# Original formula used by Paul:}
  \CommentTok{\# Unifrac \textasciitilde{} Voyage counts  + env similarity + ecoregion + random port effects}
  \KeywordTok{as.formula}\NormalTok{(RESP\_UNIFRAC }\OperatorTok{\textasciitilde{}}\StringTok{ }\NormalTok{PRED\_ENV }\OperatorTok{+}\StringTok{ }\NormalTok{ECO\_DIFF }\OperatorTok{+}\StringTok{ }\NormalTok{(}\DecValTok{1} \OperatorTok{|}\StringTok{ }\NormalTok{PORT) }\OperatorTok{+}\StringTok{ }\NormalTok{(}\DecValTok{1} \OperatorTok{|}\StringTok{ }\NormalTok{DEST)),}

  \CommentTok{\# Original formula adjusted with Mandana\textquotesingle{}s frequencies:}
  \CommentTok{\# Unifrac \textasciitilde{} Voyage frequencies + env similarity + ecoregion + random port effects}
  \KeywordTok{as.formula}\NormalTok{(RESP\_UNIFRAC }\OperatorTok{\textasciitilde{}}\StringTok{ }\NormalTok{PRED\_ENV }\OperatorTok{+}\StringTok{ }\NormalTok{ECO\_DIFF }\OperatorTok{+}\StringTok{ }\NormalTok{(}\DecValTok{1} \OperatorTok{|}\StringTok{ }\NormalTok{PORT) }\OperatorTok{+}\StringTok{ }\NormalTok{(}\DecValTok{1} \OperatorTok{|}\StringTok{ }\NormalTok{DEST)),}

  \CommentTok{\# Formulas from Word document with Mandana\textquotesingle{}s ballast risk estimates:}
  \CommentTok{\# Unifrac \textasciitilde{} Ballast FON shipping + env similarity + ecoregion + random port effects}
  \KeywordTok{as.formula}\NormalTok{(RESP\_UNIFRAC }\OperatorTok{\textasciitilde{}}\StringTok{ }\NormalTok{PRED\_ENV }\OperatorTok{+}\StringTok{ }\NormalTok{ECO\_DIFF }\OperatorTok{+}\StringTok{ }\NormalTok{(}\DecValTok{1} \OperatorTok{|}\StringTok{ }\NormalTok{PORT) }\OperatorTok{+}\StringTok{ }\NormalTok{(}\DecValTok{1} \OperatorTok{|}\StringTok{ }\NormalTok{DEST)),}

  \CommentTok{\# Unifrac \textasciitilde{} Ballast HON shipping + env similarity + ecoregion + random port effects}
  \KeywordTok{as.formula}\NormalTok{(RESP\_UNIFRAC }\OperatorTok{\textasciitilde{}}\StringTok{ }\NormalTok{PRED\_ENV }\OperatorTok{+}\StringTok{ }\NormalTok{ECO\_DIFF }\OperatorTok{+}\StringTok{ }\NormalTok{(}\DecValTok{1} \OperatorTok{|}\StringTok{ }\NormalTok{PORT) }\OperatorTok{+}\StringTok{ }\NormalTok{(}\DecValTok{1} \OperatorTok{|}\StringTok{ }\NormalTok{DEST)),}

  \CommentTok{\# Unifrac \textasciitilde{} Ballast FON risk* + \textasciitilde{}\textasciitilde{}ecoregion\textasciitilde{}\textasciitilde{} / env similarity + random port effects}
  \KeywordTok{as.formula}\NormalTok{(RESP\_UNIFRAC }\OperatorTok{\textasciitilde{}}\StringTok{ }\NormalTok{PRED\_ENV }\OperatorTok{+}\StringTok{ }\NormalTok{(}\DecValTok{1} \OperatorTok{|}\StringTok{ }\NormalTok{PORT) }\OperatorTok{+}\StringTok{ }\NormalTok{(}\DecValTok{1} \OperatorTok{|}\StringTok{ }\NormalTok{DEST)),}

  \CommentTok{\# Unifrac \textasciitilde{} Ballast HON risk* + \textasciitilde{}\textasciitilde{}ecoregion\textasciitilde{}\textasciitilde{} / env similarity + random port effects}
  \KeywordTok{as.formula}\NormalTok{(RESP\_UNIFRAC }\OperatorTok{\textasciitilde{}}\StringTok{ }\NormalTok{B\_HON\_SMECO }\OperatorTok{+}\StringTok{ }\NormalTok{PRED\_ENV }\OperatorTok{+}\StringTok{ }\NormalTok{(}\DecValTok{1} \OperatorTok{|}\StringTok{ }\NormalTok{PORT) }\OperatorTok{+}\StringTok{ }\NormalTok{(}\DecValTok{1} \OperatorTok{|}\StringTok{ }\NormalTok{DEST))}
\NormalTok{)}
\end{Highlighting}
\end{Shaded}

\hypertarget{read-in-and-format-data}{%
\section{Read in and format data}\label{read-in-and-format-data}}

Please refer to project README.md file for further details on previous
processing steps (dated 31-Jan-2020).

\begin{Shaded}
\begin{Highlighting}[]
\CommentTok{\# define file path components for listing }
\NormalTok{model\_input\_folder <{-}}\StringTok{ "/Users/paul/Documents/CU\_combined/Zenodo/Results"}
\NormalTok{model\_input\_pattern <{-}}\StringTok{ }\KeywordTok{glob2rx}\NormalTok{(}\StringTok{"??\_results\_euk\_*\_model\_data\_*.csv"}\NormalTok{)}

\CommentTok{\# read all file into lists for \textasciigrave{}lapply()\textasciigrave{} usage}
\NormalTok{model\_input\_files <{-}}\StringTok{ }\KeywordTok{list.files}\NormalTok{(}\DataTypeTok{path=}\NormalTok{model\_input\_folder, }
  \DataTypeTok{pattern =}\NormalTok{ model\_input\_pattern, }\DataTypeTok{full.names =} \OtherTok{TRUE}\NormalTok{)}

\CommentTok{\# store all tables in list and save input filenames alongside {-} skipping "X1" }
\CommentTok{\#  in case previous tables have column numbers, which they should not have anymore.}
\NormalTok{model\_input\_data <{-}}\StringTok{ }\KeywordTok{suppressWarnings}\NormalTok{(}\KeywordTok{lapply}\NormalTok{(model\_input\_files, }
  \ControlFlowTok{function}\NormalTok{(listed\_file)  }\KeywordTok{read\_csv}\NormalTok{(listed\_file, }\DataTypeTok{col\_types =} \KeywordTok{cols}\NormalTok{(}\StringTok{\textquotesingle{}X1\textquotesingle{}}\NormalTok{ =}\StringTok{ }\KeywordTok{col\_skip}\NormalTok{()))))}
\KeywordTok{names}\NormalTok{(model\_input\_data) <{-}}\StringTok{ }\NormalTok{model\_input\_files}
\end{Highlighting}
\end{Shaded}

\hypertarget{modelling}{%
\section{Modelling}\label{modelling}}

\hypertarget{get-full-models}{%
\subsection{Get full models}\label{get-full-models}}

\begin{Shaded}
\begin{Highlighting}[]
\CommentTok{\# loop over input tables and models}

\CommentTok{\# full\_models}



\CommentTok{\# flatten list}
\end{Highlighting}
\end{Shaded}

\hypertarget{get-anovas}{%
\subsection{Get ANOVAs}\label{get-anovas}}

\hypertarget{check-aics}{%
\subsection{Check AICs}\label{check-aics}}

\hypertarget{plot-model-results}{%
\subsection{Plot model results}\label{plot-model-results}}

\begin{Shaded}
\begin{Highlighting}[]
\CommentTok{\# lmer( full\_models[[1]], data = model\_input\_data[[1]], REML=FALSE)}
\CommentTok{\# }
\CommentTok{\# }
\CommentTok{\# \#\textquotesingle{} \# Define null models}
\CommentTok{\# }
\CommentTok{\# }
\CommentTok{\# \# }
\CommentTok{\# }
\CommentTok{\# \# anova using mapply over lists}
\CommentTok{\# }
\CommentTok{\# }
\CommentTok{\# \#\textquotesingle{} For reasons of simplicity loop over list to generate model results and plots for each inout data set.}
\CommentTok{\# }
\CommentTok{\# \#\textquotesingle{}}
\CommentTok{\# \#\textquotesingle{} \# Apply models to list of formatted input tables }
\CommentTok{\# \#\textquotesingle{} }
\CommentTok{\# \#\textquotesingle{} \#\# Model formulas: }
\CommentTok{\# \#\textquotesingle{}}
\CommentTok{\# \#\textquotesingle{} }\AlertTok{\#\#\#}\CommentTok{ Original formula used by Paul:}
\CommentTok{\# \#\textquotesingle{} }
\CommentTok{\# \#\textquotesingle{} 0. \textasciigrave{} Unifrac \textasciitilde{} Voyage counts  + env similarity + ecoregion + random port effects\textasciigrave{}:}
\CommentTok{\# }
\CommentTok{\# \# full\_model <{-} lmer(RESP\_UNIFRAC \textasciitilde{} PRED\_TRIPS + PRED\_ENV + ECO\_DIFF + (1 | PORT) + (1 | DEST), data = vars, REML=FALSE)}
\CommentTok{\# }
\CommentTok{\# \#\textquotesingle{} }\AlertTok{\#\#\#}\CommentTok{ Original formula adjusted with Mandana\textquotesingle{}s frequencies:}
\CommentTok{\# \#\textquotesingle{}}
\CommentTok{\# \#\textquotesingle{} 1. \textasciigrave{}Unifrac \textasciitilde{} Voyage frequencies + env similarity + ecoregion + random port effects\textasciigrave{}:}
\CommentTok{\# }
\CommentTok{\# \# full\_model <{-} lmer(RESP\_UNIFRAC \textasciitilde{} VOY\_FREQ + PRED\_ENV + ECO\_DIFF + (1 | PORT) + (1 | DEST), data = vars, REML=FALSE)}
\CommentTok{\# }
\CommentTok{\# \#\textquotesingle{} }\AlertTok{\#\#\#}\CommentTok{ Formulas from Word document with Mandana\textquotesingle{}s ballast risk estimates:}
\CommentTok{\# \#\textquotesingle{} }
\CommentTok{\# \#\textquotesingle{} 2.  \textasciigrave{}Unifrac \textasciitilde{} Ballast FON shipping + env similarity + ecoregion + random port effects\textasciigrave{}:}
\CommentTok{\# }
\CommentTok{\# \# full\_model <{-} lmer(RESP\_UNIFRAC \textasciitilde{} B\_FON\_NOECO + PRED\_ENV + ECO\_DIFF + (1 | PORT) + (1 | DEST), data = vars, REML=FALSE)}
\CommentTok{\# }
\CommentTok{\# \#\textquotesingle{} 3.  \textasciigrave{}Unifrac \textasciitilde{} Ballast HON shipping + env similarity + ecoregion + random port effects\textasciigrave{}:}
\CommentTok{\# }
\CommentTok{\# \# full\_model <{-} lmer(RESP\_UNIFRAC \textasciitilde{} B\_HON\_NOECO + PRED\_ENV + ECO\_DIFF + (1 | PORT) + (1 | DEST), data = vars, REML=FALSE)}
\CommentTok{\# }
\CommentTok{\# \#\textquotesingle{} 4.  \textasciigrave{}Unifrac \textasciitilde{} Ballast FON risk* + \textasciitilde{}\textasciitilde{}ecoregion\textasciitilde{}\textasciitilde{} / env similarity + random port effects\textasciigrave{} (Ecoregion from Word documents substituted with Environmental Similarity since Mandana considers Ecoregions?):}
\CommentTok{\# }
\CommentTok{\# \# full\_model <{-} lmer(RESP\_UNIFRAC \textasciitilde{} B\_FON\_SMECO + PRED\_ENV + (1 | PORT) + (1 | DEST), data = vars, REML=FALSE)}
\CommentTok{\# }
\CommentTok{\# \#\textquotesingle{} 5.  \textasciigrave{}Unifrac \textasciitilde{} Ballast HON risk* + \textasciitilde{}\textasciitilde{}ecoregion\textasciitilde{}\textasciitilde{} / env similarity + random port effects\textasciigrave{} (Ecoregion from Word documents substituted with Environmental Similarity since Mandana considers Ecoregions?):}
\CommentTok{\# }
\CommentTok{\# \# full\_model <{-} lmer(RESP\_UNIFRAC \textasciitilde{} B\_HON\_SMECO + PRED\_ENV + (1 | PORT) + (1 | DEST), data = vars, REML=FALSE)}
\CommentTok{\# }
\CommentTok{\# \#\textquotesingle{}}
\CommentTok{\# }
\CommentTok{\# \# {-}{-}{-}{-}{-}{-}{-}{-}{-}{-}{-}{-}{-}{-}{-}{-}{-}{-}{-}{-}{-}{-}{-}{-}{-} old code below {-}{-}{-}{-}{-}{-}{-}{-}{-}{-}{-}{-}{-}{-}{-}{-}{-}{-}{-}{-}{-}{-}}
\CommentTok{\# \# }
\CommentTok{\# \# Checking response and predictor variable distributions}
\CommentTok{\# \# }
\CommentTok{\# \# \# Plots {-}  create list to store plots for export further below}
\CommentTok{\# \# }
\CommentTok{\# \# \#\# Responses}
\CommentTok{\# \# plot\_l <{-} list()}
\CommentTok{\# \# }
\CommentTok{\# \# plot\_l[[1]] <{-} ggplot(model\_data, aes (x=RESP\_UNIFRAC)) + }
\CommentTok{\# \#               geom\_density() +}
\CommentTok{\# \#               facet\_grid(\textasciitilde{}ECO\_DIFF) +}
\CommentTok{\# \#               theme\_bw()}
\CommentTok{\# \# }
\CommentTok{\# \# plot\_l[[2]] <{-} ggplot(model\_data,aes (x=PRED\_ENV))+ }
\CommentTok{\# \#               geom\_density()+}
\CommentTok{\# \#               facet\_grid(\textasciitilde{}ECO\_DIFF)+}
\CommentTok{\# \#               theme\_bw()}
\CommentTok{\# \# }
\CommentTok{\# \# }
\CommentTok{\# \# \#\# Pedictors}
\CommentTok{\# \# }
\CommentTok{\# \# plot\_l[[3]] <{-}  ggplot(model\_data,aes (x=PRED\_TRIPS))+ }
\CommentTok{\# \#                geom\_density()+}
\CommentTok{\# \#                facet\_grid(\textasciitilde{}ECO\_DIFF)+}
\CommentTok{\# \#                theme\_bw()}
\CommentTok{\# \# }
\CommentTok{\# \# plots <{-} plot\_grid(plot\_l[[1]], plot\_l[[2]], plot\_l[[3]],}
\CommentTok{\# \#           labels=c("Eco(T/F)", "Eco(T/F)", "Eco(T/F)" ))}
\CommentTok{\# \# }
\CommentTok{\# \# save\_plot(args[5], plots, ncol = 1, nrow = 1, base\_height = 5,}
\CommentTok{\# \#   base\_aspect\_ratio = 1.1, base\_width = 10)}
\CommentTok{\# \#   }
\CommentTok{\# \# }
\CommentTok{\# \# Appending to output file.}
\CommentTok{\# \# capture.output( file=args[6], append=TRUE, print("Aggregation of biological distance means per ecoregion. (Response variable)"))}
\CommentTok{\# \# capture.output( file=args[6], append=TRUE, aggregate(model\_data$RESP\_UNIFRAC\textasciitilde{}model\_data$ECO\_DIFF, FUN=mean))}
\CommentTok{\# \# }
\CommentTok{\# \# capture.output( file=args[6], append=TRUE, print("Aggregation of environmental distance means per ecoregion (Predictor variable)."))}
\CommentTok{\# \# capture.output( file=args[6], append=TRUE, aggregate(model\_data$PRED\_ENV\textasciitilde{}model\_data$ECO\_DIFF, FUN=mean))}
\CommentTok{\# \# }
\CommentTok{\# \# capture.output( file=args[6], append=TRUE, print("Aggregation of summed voyage counts means per ecoregion (Predictor variable)."))}
\CommentTok{\# \# capture.output( file=args[6], append=TRUE, aggregate(model\_data$PRED\_TRIPS\textasciitilde{}model\_data$ECO\_DIFF, FUN=mean))}
\CommentTok{\# \# }
\CommentTok{\# \# \textquotesingle{}}
\CommentTok{\# \# \textquotesingle{} <!{-}{-} \#\#\#\#\#\#\#\#\#\#\#\#\#\#\#\#\#\#\#\#\#\#\#\#\#\#\#\#\#\#\#\#\#\#\#\#\#\#\#\#\#\#\#\#\#\#\#\#\#\#\#\#\#\#\#\#\#\#\#\#\#\#\#\#\#\#\#\# {-}{-}>}
\CommentTok{\# \# \textquotesingle{}}
\CommentTok{\# \# \textquotesingle{} <!{-}{-} \#\#\#\#\#\#\#\#\#\#\#\#\#\#\#\#\#\#\#\#\#\#\#\#\#\#\#\#\#\#\#\#\#\#\#\#\#\#\#\#\#\#\#\#\#\#\#\#\#\#\#\#\#\#\#\#\#\#\#\#\#\#\#\#\#\#\#\# {-}{-}>}
\CommentTok{\# \# \textquotesingle{}}
\CommentTok{\# \# }
\CommentTok{\# \# model\_data <{-} model\_data \%>\%  mutate\_if(is.numeric, scale(.,scale = FALSE))}
\CommentTok{\# \# pairs(RESP\_UNIFRAC \textasciitilde{} PRED\_ENV * PRED\_TRIPS, data=model\_data, main="Simple Scatterplot Matrix")}
\CommentTok{\# \# }
\CommentTok{\# \# \textquotesingle{} \# Select variables for modelling and build models }
\CommentTok{\# \# message("Saving this model data to file:")}
\CommentTok{\# \# print(model\_data)}
\CommentTok{\# \# }
\CommentTok{\# \# Sorting columns}
\CommentTok{\# \# model\_data <{-} model\_data \%>\% arrange(PORT, desc(PRED\_TRIPS), DEST)}
\CommentTok{\# \# }
\CommentTok{\# \# correcting trips for Pearl Harbour}
\CommentTok{\# \# model\_data <{-} model\_data \%>\% mutate (PRED\_TRIPS = ifelse(PORT  == "PH", "0", PRED\_TRIPS))}
\CommentTok{\# \# model\_data <{-} model\_data \%>\% mutate (PRED\_TRIPS = ifelse(DEST  == "PH", "0", PRED\_TRIPS))}
\CommentTok{\# \# }
\CommentTok{\# \# model\_data$PORT <{-} as.factor(model\_data$PORT)}
\CommentTok{\# \# model\_data$DEST <{-} as.factor(model\_data$DEST)}
\CommentTok{\# \# model\_data$ECO\_DIFF <{-} as.factor(model\_data$ECO\_DIFF)}
\CommentTok{\# \# model\_data$PRED\_TRIPS <{-} as.numeric(model\_data$PRED\_TRIPS)}
\CommentTok{\# \# }
\CommentTok{\# \# }
\CommentTok{\# \# write data as per input path {-} keep close to variable selection below}
\CommentTok{\# \# write.csv(model\_data, file = args[4])}
\CommentTok{\# \# }
\CommentTok{\# \# select  columns for model}
\CommentTok{\# \# vars <{-} model\_data \%>\% select(RESP\_UNIFRAC, PORT, DEST, ECO\_DIFF, PRED\_ENV, PRED\_TRIPS)}
\CommentTok{\# \# message("Using this model data for regression:")}
\CommentTok{\# \# print(vars)}
\CommentTok{\# \# }
\CommentTok{\# \# \textquotesingle{} \#\# Full Model and checking }
\CommentTok{\# \# vars\_model\_full <{-} lmer(RESP\_UNIFRAC \textasciitilde{} PRED\_ENV + PRED\_TRIPS + ECO\_DIFF + (1 | PORT) + (1 | DEST), data=vars, REML=FALSE)}
\CommentTok{\# \# }
\CommentTok{\# \# vars\_model\_full <{-} lm(RESP\_UNIFRAC \textasciitilde{} PRED\_ENV + PRED\_TRIPS + ECO\_DIFF, data=vars)}
\CommentTok{\# \# }
\CommentTok{\# \# \textquotesingle{} }\AlertTok{\#\#\#}\CommentTok{ Model Summary}
\CommentTok{\# \# }
\CommentTok{\# \# Appending to output file.}
\CommentTok{\# \# capture.output( file=args[6], append=TRUE,   message("Considered vraiables"))}
\CommentTok{\# \# capture.output( file=args[6], append=TRUE,   print(vars\_model\_full))}
\CommentTok{\# \#   }
\CommentTok{\# \# capture.output( file=args[6], append=TRUE,   message("Random effect model summary"))}
\CommentTok{\# \# capture.output( file=args[6], append=TRUE,   print(summary(vars\_model\_full)))}
\CommentTok{\# \#   }
\CommentTok{\# \# capture.output( file=args[6], append=TRUE,  message("Intercepts for factor levels"))}
\CommentTok{\# \# capture.output( file=args[6], append=TRUE,  print(coef(vars\_model\_full))) \#intercept for each level}
\CommentTok{\# \# }
\CommentTok{\# \# For linear models, you can also plot standardized beta coefficients,}
\CommentTok{\# \# https://cran.r{-}project.org/web/packages/sjPlot/vignettes/plot\_model\_estimates.html}
\CommentTok{\# \# using type = "std" or type = "std2". These two options differ in the way how}
\CommentTok{\# \# coefficients are standardized. type = "std2" plots standardized beta values,}
\CommentTok{\# \# however, standardization follows Gelman’s (2008) suggestion, }
\CommentTok{\# \# rescaling the estimates by dividing them by two standard deviations}
\CommentTok{\# \# instead of just one. (Use, type = std)}
\CommentTok{\# \# }
\CommentTok{\# \# p <{-} plot\_model(vars\_model\_full, show.values = TRUE, value.offset = .3,}
\CommentTok{\# \#    type = "std", }
\CommentTok{\# \#    title = args[8])}
\CommentTok{\# \# }
\CommentTok{\# \# save\_plot(args[7], p, ncol = 1, nrow = 1, base\_height = 8,}
\CommentTok{\# \#   base\_aspect\_ratio = 1.1, base\_width = 8)}
\CommentTok{\# \# }
\CommentTok{\# \# \textquotesingle{} Residuals}
\CommentTok{\# \# plot(vars\_model\_full)}
\CommentTok{\# \# }
\CommentTok{\# \# \# check normality of the residuals}
\CommentTok{\# \# qqnorm(residuals(vars\_model\_full))}
\CommentTok{\# \# }
\CommentTok{\# \# \# plotting random effects {-} trial after https://stackoverflow.com/questions/13847936/in{-}r{-}plotting{-}random{-}effects{-}from{-}lmer{-}lme4{-}package{-}using{-}qqmath{-}or{-}dotplot}
\CommentTok{\# \# qqplot of the random effects with their variances}
\CommentTok{\# \# "The last line of code produces a really nice plot of each intercept with the error around each estimate."}
\CommentTok{\# \# qqmath(ranef(vars\_model\_full, condVar = TRUE), strip = FALSE)$PORT}
\CommentTok{\# \# qqmath(ranef(vars\_model\_full, condVar = TRUE), strip = FALSE)$DEST}
\CommentTok{\# \# }
\CommentTok{\# \# }
\CommentTok{\# \# \textquotesingle{} }\AlertTok{\#\#\#}\CommentTok{ Leverage of Observations}
\CommentTok{\# \# }
\CommentTok{\# \# \# model is not influenced by one or a small set of observations ?}
\CommentTok{\# \# ggplot(data.frame(lev=hatvalues(vars\_model\_full),pearson=residuals(vars\_model\_full, type="pearson")),}
\CommentTok{\# \#       aes(x=lev,y=pearson)) + geom\_point() + theme\_bw()}
\CommentTok{\# \# }
\CommentTok{\# \# \textquotesingle{} \#\# Null Model and checking }
\CommentTok{\# \# }
\CommentTok{\# \# vars\_model\_null <{-} lmer(RESP\_UNIFRAC \textasciitilde{} PRED\_ENV + ECO\_DIFF + (1 | PORT) + (1 | DEST), data=vars, REML=FALSE)}
\CommentTok{\# \# }
\CommentTok{\# \# vars\_model\_null <{-} lm(RESP\_UNIFRAC \textasciitilde{} PRED\_ENV + ECO\_DIFF, data=vars)}
\CommentTok{\# \# }
\CommentTok{\# \# \textquotesingle{} }\AlertTok{\#\#\#}\CommentTok{ Model Summary}
\CommentTok{\# \# summary(vars\_model\_null)}
\CommentTok{\# \# }
\CommentTok{\# \# Appending to output file.}
\CommentTok{\# \# \textquotesingle{} \#\# Model Significance}
\CommentTok{\# \# capture.output( file=args[6],   anova(vars\_model\_null, vars\_model\_full))}
\CommentTok{\# }
\CommentTok{\# \#\textquotesingle{} <!{-}{-} \#\#\#\#\#\#\#\#\#\#\#\#\#\#\#\#\#\#\#\#\#\#\#\#\#\#\#\#\#\#\#\#\#\#\#\#\#\#\#\#\#\#\#\#\#\#\#\#\#\#\#\#\#\#\#\#\#\#\#\#\#\#\#\#\#\#\#\# {-}{-}>}
\CommentTok{\# \#\textquotesingle{}}
\CommentTok{\# \#\textquotesingle{} \# Session info}
\CommentTok{\# \#\textquotesingle{}}
\CommentTok{\# \#\textquotesingle{} The code and output in this document were tested and generated in the}
\CommentTok{\# \#\textquotesingle{} following computing environment:}
\CommentTok{\# \#+ echo=FALSE}
\CommentTok{\# sessionInfo()}
\CommentTok{\# }
\CommentTok{\# \#\textquotesingle{} \# References}
\end{Highlighting}
\end{Shaded}


\end{document}
